%% bare/jrnl.tex
%% V1.4b
%% 2015/08/26
%% by Michael  Shell
%% see http://www.michaelshell.org/
%% for current contact information.
%%
%% This is a skeleton file demonstrating the use of IEEEtran.cls
%% (requires IEEEtran.cls version 1.8b or later) with an IEEE
%% journal paper.
%%
%% Support sites:
%% http://www.michaelshell.org/tex/ieeetran/
%% http://www.ctan.org/pkg/ieeetran
%% and
%% http://www.ieee.org/

%%*************************************************************************
%% Legal Notice:
%% This code is offered as-is without any warranty either expressed or
%% implied; without even the implied warranty of MERCHANTABILITY or
%% FITNESS FOR A PARTICULAR PURPOSE! 
%% User assumes all risk.
%% In no event shall the IEEE or any contributor to this code be liable for
%% any damages or losses, including, but not limited to, incidental,
%% consequential, or any other damages, resulting from the use or misuse
%% of any information contained here.
%%
%% All comments are the opinions of their respective authors and are not
%% necessarily endorsed by the IEEE.
%%
%% This work is distributed under the LaTeX Project Public License (LPPL)
%% ( http://www.latex-project.org/ ) version 1.3, and may be freely used,
%% distributed and modified. A copy of the LPPL, version 1.3, is included
%% in the base LaTeX documentation of all distributions of LaTeX released
%% 2003/12/01 or later.
%% Retain all contribution notices and credits.
%% ** Modified files should be clearly indicated as such, including  **
%% ** renaming them and changing author support contact information. **
%%*************************************************************************


% *** Authors should verify (and, if needed, correct) their LaTeX system  ***
% *** with the testflow diagnostic prior to trusting their LaTeX platform ***
% *** with production work. The IEEE's font choices and paper sizes can   ***
% *** trigger bugs that do not appear when using other class files.       ***                          ***
% The testflow support page is at:
% http://www.michaelshell.org/tex/testflow/



\documentclass[journal]{IEEEtran}
%
% If IEEEtran.cls has not been installed into the LaTeX system files,
% manually specify the path to it like:
% \documentclass[journal]{../sty/IEEEtran}





% Some very useful LaTeX packages include:
% (uncomment the ones you want to load)


% *** MISC UTILITY PACKAGES ***
%
%\usepackage{ifpdf}
% Heiko Oberdiek's ifpdf.sty is very useful if you need conditional
% compilation based on whether the output is pdf or dvi.
% usage:
% \ifpdf
%   % pdf code
% \else
%   % dvi code
% \fi
% The latest version of ifpdf.sty can be obtained from:
% http://www.ctan.org/pkg/ifpdf
% Also, note that IEEEtran.cls V1.7 and later provides a builtin
% \ifCLASSINFOpdf conditional that works the same way.
% When switching from latex to pdflatex and vice-versa, the compiler may
% have to be run twice to clear warning/error messages.






% *** CITATION PACKAGES ***
%
%\usepackage{cite}
% cite.sty was written by Donald Arseneau
% V1.6 and later of IEEEtran pre-defines the format of the cite.sty package
% \cite{} output to follow that of the IEEE. Loading the cite package will
% result in citation numbers being automatically sorted and properly
% "compressed/ranged". e.g., [1], [9], [2], [7], [5], [6] without using
% cite.sty will become [1], [2], [5]--[7], [9] using cite.sty. cite.sty's
% \cite will automatically add leading space, if needed. Use cite.sty's
% noadjust option (cite.sty V3.8 and later) if you want to turn this off
% such as if a citation ever needs to be enclosed in parenthesis.
% cite.sty is already installed on most LaTeX systems. Be sure and use
% version 5.0 (2009-03-20) and later if using hyperref.sty.
% The latest version can be obtained at:
% http://www.ctan.org/pkg/cite
% The documentation is contained in the cite.sty file itself.






% *** GRAPHICS RELATED PACKAGES ***
%
\ifCLASSINFOpdf
  % \usepackage[pdftex]{graphicx}
  % declare the path(s) where your graphic files are
  % \graphicspath{{../pdf/}{../jpeg/}}
  % and their extensions so you won't have to specify these with
  % every instance of \includegraphics
  % \DeclareGraphicsExtensions{.pdf,.jpeg,.png}
\else
  % or other class option (dvipsone, dvipdf, if not using dvips). graphicx
  % will default to the driver specified in the system graphics.cfg if no
  % driver is specified.
  % \usepackage[dvips]{graphicx}
  % declare the path(s) where your graphic files are
  % \graphicspath{{../eps/}}
  % and their extensions so you won't have to specify these with
  % every instance of \includegraphics
  % \DeclareGraphicsExtensions{.eps}
\fi
% graphicx was written by David Carlisle and Sebastian Rahtz. It is
% required if you want graphics, photos, etc. graphicx.sty is already
% installed on most LaTeX systems. The latest version and documentation
% can be obtained at: 
% http://www.ctan.org/pkg/graphicx
% Another good source of documentation is "Using Imported Graphics in
% LaTeX2e" by Keith Reckdahl which can be found at:
% http://www.ctan.org/pkg/epslatex
%
% latex, and pdflatex in dvi mode, support graphics in encapsulated
% postscript (.eps) format. pdflatex in pdf mode supports graphics
% in .pdf, .jpeg, .png and .mps (metapost) formats. Users should ensure
% that all non-photo figures use a vector format (.eps, .pdf, .mps) and
% not a bitmapped formats (.jpeg, .png). The IEEE frowns on bitmapped formats
% which can result in "jaggedy"/blurry rendering of lines and letters as
% well as large increases in file sizes.
%
% You can find documentation about the pdfTeX application at:
% http://www.tug.org/applications/pdftex





% *** MATH PACKAGES ***
%
%\usepackage{amsmath}
% A popular package from the American Mathematical Society that provides
% many useful and powerful commands for dealing with mathematics.
%
% Note that the amsmath package sets \interdisplaylinepenalty to 10000
% thus preventing page breaks from occurring within multiline equations. Use:
%\interdisplaylinepenalty=2500
% after loading amsmath to restore such page breaks as IEEEtran.cls normally
% does. amsmath.sty is already installed on most LaTeX systems. The latest
% version and documentation can be obtained at:
% http://www.ctan.org/pkg/amsmath





% *** SPECIALIZED LIST PACKAGES ***
%
%\usepackage{algorithmic}
% algorithmic.sty was written by Peter Williams and Rogerio Brito.
% This package provides an algorithmic environment fo describing algorithms.
% You can use the algorithmic environment in-text or within a figure
% environment to provide for a floating algorithm. Do NOT use the algorithm
% floating environment provided by algorithm.sty (by the same authors) or
% algorithm2e.sty (by Christophe Fiorio) as the IEEE does not use dedicated
% algorithm float types and packages that provide these will not provide
% correct IEEE style captions. The latest version and documentation of
% algorithmic.sty can be obtained at:
% http://www.ctan.org/pkg/algorithms
% Also of interest may be the (relatively newer and more customizable)
% algorithmicx.sty package by Szasz Janos:
% http://www.ctan.org/pkg/algorithmicx




% *** ALIGNMENT PACKAGES ***
%
%\usepackage{array}
% Frank Mittelbach's and David Carlisle's array.sty patches and improves
% the standard LaTeX2e array and tabular environments to provide better
% appearance and additional user controls. As the default LaTeX2e table
% generation code is lacking to the point of almost being broken with
% respect to the quality of the end results, all users are strongly
% advised to use an enhanced (at the very least that provided by array.sty)
% set of table tools. array.sty is already installed on most systems. The
% latest version and documentation can be obtained at:
% http://www.ctan.org/pkg/array


% IEEEtran contains the IEEEeqnarray family of commands that can be used to
% generate multiline equations as well as matrices, tables, etc., of high
% quality.




% *** SUBFIGURE PACKAGES ***
\ifCLASSOPTIONcompsoc
 \usepackage[caption=false,font=normalsize,labelfont=sf,textfont=sf]{subfig}
\else
 \usepackage[caption=false,font=footnotesize]{subfig}
\fi
% subfig.sty, written by Steven Douglas Cochran, is the modern replacement
% for subfigure.sty, the latter of which is no longer maintained and is
% incompatible with some LaTeX packages including fixltx2e. However,
% subfig.sty requires and automatically loads Axel Sommerfeldt's caption.sty
% which will override IEEEtran.cls' handling of captions and this will result
% in non-IEEE style figure/table captions. To prevent this problem, be sure
% and invoke subfig.sty's "caption=false" package option (available since
% subfig.sty version 1.3, 2005/06/28) as this is will preserve IEEEtran.cls
% handling of captions.
% Note that the Computer Society format requires a larger sans serif font
% than the serif footnote size font used in traditional IEEE formatting
% and thus the need to invoke different subfig.sty package options depending
% on whether compsoc mode has been enabled.
%
% The latest version and documentation of subfig.sty can be obtained at:
% http://www.ctan.org/pkg/subfig




% *** FLOAT PACKAGES ***
%
%\usepackage{fixltx2e}
% fixltx2e, the successor to the earlier fix2col.sty, was written by
% Frank Mittelbach and David Carlisle. This package corrects a few problems
% in the LaTeX2e kernel, the most notable of which is that in current
% LaTeX2e releases, the ordering of single and double column floats is not
% guaranteed to be preserved. Thus, an unpatched LaTeX2e can allow a
% single column figure to be placed prior to an earlier double column
% figure.
% Be aware that LaTeX2e kernels dated 2015 and later have fixltx2e.sty's
% corrections already built into the system in which case a warning will
% be issued if an attempt is made to load fixltx2e.sty as it is no longer
% needed.
% The latest version and documentation can be found at:
% http://www.ctan.org/pkg/fixltx2e


%\usepackage{stfloats}
% stfloats.sty was written by Sigitas Tolusis. This package gives LaTeX2e
% the ability to do double column floats at the bottom of the page as well
% as the top. (e.g., "\begin{figure*}[!b]" is not normally possible in
% LaTeX2e). It also provides a command:
%\fnbelowfloat
% to enable the placement of footnotes below bottom floats (the standard
% LaTeX2e kernel puts them above bottom floats). This is an invasive package
% which rewrites many portions of the LaTeX2e float routines. It may not work
% with other packages that modify the LaTeX2e float routines. The latest
% version and documentation can be obtained at:
% http://www.ctan.org/pkg/stfloats
% Do not use the stfloats baselinefloat ability as the IEEE does not allow
% \baselineskip to stretch. Authors submitting work to the IEEE should note
% that the IEEE rarely uses double column equations and that authors should try
% to avoid such use. Do not be tempted to use the cuted.sty or midfloat.sty
% packages (also by Sigitas Tolusis) as the IEEE does not format its papers in
% such ways.
% Do not attempt to use stfloats with fixltx2e as they are incompatible.
% Instead, use Morten Hogholm'a dblfloatfix which combines the features
% of both fixltx2e and stfloats:
%
% \usepackage{dblfloatfix}
% The latest version can be found at:
% http://www.ctan.org/pkg/dblfloatfix




%\ifCLASSOPTIONcaptionsoff
%  \usepackage[nomarkers]{endfloat}
% \let\MYoriglatexcaption\caption
% \renewcommand{\caption}[2][\relax]{\MYoriglatexcaption[#2]{#2}}
%\fi
% endfloat.sty was written by James Darrell McCauley, Jeff Goldberg and 
% Axel Sommerfeldt. This package may be useful when used in conjunction with 
% IEEEtran.cls'  captionsoff option. Some IEEE journals/societies require that
% submissions have lists of figures/tables at the end of the paper and that
% figures/tables without any captions are placed on a page by themselves at
% the end of the document. If needed, the draftcls IEEEtran class option or
% \CLASSINPUTbaselinestretch interface can be used to increase the line
% spacing as well. Be sure and use the nomarkers option of endfloat to
% prevent endfloat from "marking" where the figures would have been placed
% in the text. The two hack lines of code above are a slight modification of
% that suggested by in the endfloat docs (section 8.4.1) to ensure that
% the full captions always appear in the list of figures/tables - even if
% the user used the short optional argument of \caption[]{}.
% IEEE papers do not typically make use of \caption[]'s optional argument,
% so this should not be an issue. A similar trick can be used to disable
% captions of packages such as subfig.sty that lack options to turn off
% the subcaptions:
% For subfig.sty:
% \let\MYorigsubfloat\subfloat
% \renewcommand{\subfloat}[2][\relax]{\MYorigsubfloat[]{#2}}
% However, the above trick will not work if both optional arguments of
% the \subfloat command are used. Furthermore, there needs to be a
% description of each subfigure *somewhere* and endfloat does not add
% subfigure captions to its list of figures. Thus, the best approach is to
% avoid the use of subfigure captions (many IEEE journals avoid them anyway)
% and instead reference/explain all the subfigures within the main caption.
% The latest version of endfloat.sty and its documentation can obtained at:
% http://www.ctan.org/pkg/endfloat
%
% The IEEEtran \ifCLASSOPTIONcaptionsoff conditional can also be used
% later in the document, say, to conditionally put the References on a 
% page by themselves.




% *** PDF, URL AND HYPERLINK PACKAGES ***
%
%\usepackage{url}
% url.sty was written by Donald Arseneau. It provides better support for
% handling and breaking URLs. url.sty is already installed on most LaTeX
% systems. The latest version and documentation can be obtained at:
% http://www.ctan.org/pkg/url
% Basically, \url{my/url/here}.




% *** Do not adjust lengths that control margins, column widths, etc. ***
% *** Do not use packages that alter fonts (such as pslatex).         ***
% There should be no need to do such things with IEEEtran.cls V1.6 and later.
% (Unless specifically asked to do so by the journal or conference you plan
% to submit to, of course. )


% correct bad hyphenation here
\hyphenation{op-tical net-works semi-conduc-tor}

%------ Additions by Daniel
%--------------------------------------------------------------------------
%--------------------------------------------------------------------------
%for Lorem 
\usepackage{lipsum}
%for ABC lists 
% from https://tex.stackexchange.com/questions/129951/enumerate-tag-using-the-alphabet-instead-of-numbers
\usepackage{enumitem}

%Code Syntax 
\usepackage{listings}
\lstdefinelanguage{Tamarmin}
{
  % list of keywords
  keywords={},
  morekeywords={axiom, lemma, equations, functions, builtins, protocol, property, in, let, theory, begin, end, subsection, section, text, rule, aenc, sdec, senc, sdec, sign, verify, hashing, signing, multiset, \$* }
  sensitive=true, % keywords are not case-sensitive
  captionpos=b, %caption at bottom
  morecomment=[l]{//}, % l is for line comment
  morecomment=[s]{/*}{*/}, % s is for start and end delimiter
  %morestring=[b]" % defines that strings are enclosed in double quotes
  morestring=[b]', % defines that strings are enclosed in double quotes
  alsoletter=\$
}
% Define Colors
\usepackage{xcolor}
\definecolor{eclipseBlue}{RGB}{42,0.0,255}
\definecolor{eclipseGreen}{RGB}{63,127,95}
\definecolor{eclipsePurple}{RGB}{127,0,85}
 
% Set Language
\lstset{
  language={Tamarmin},
  basicstyle=\scriptsize\ttfamily, % Global Code Style
  captionpos=b, % Position of the Caption (t for top, b for bottom)
  extendedchars=true, % Allows 256 instead of 128 ASCII characters
  tabsize=4, % number of spaces indented when discovering a tab 
  columns=fixed, % make all characters equal width
  keepspaces=true, % does not ignore spaces to fit width, convert tabs to spaces
  showstringspaces=false, % lets spaces in strings appear as real spaces
  breaklines=true, % wrap lines if they don't fit
  %frame=trbl, % draw a frame at the top, right, left and bottom of the listing
  frame=tb, % draw a frame at the top and bottom of the listing
  %frameround=tttt, % make the frame round at all four corners
  framesep=4pt, % quarter circle size of the round corners
  numbers=left, % show line numbers at the left
  numberstyle=\scriptsize\ttfamily, % style of the line numbers
  commentstyle=\color{eclipseGreen}, % style of comments
  keywordstyle=\color{eclipsePurple}, % style of keywords
  stringstyle=\color{eclipseBlue}, % style of strings
}

%Commands for usign Tamarin reservered words etc in a consistent way
\usepackage{TamarinKeywords}

%For BIG letters C and M
\usepackage{ dsfont }
% for todos
\usepackage{todonotes}



%Block Diagrams
% Define block styles
\tikzstyle{decision} = [diamond, draw, fill=blue!20, 
    text width=4.5em, text badly centered, node distance=3cm, inner sep=0pt]
\tikzstyle{block} = [rectangle, draw, fill=blue!20, 
    text width=5em, text centered, rounded corners, minimum height=4em]
\tikzstyle{thickblock} = [rectangle, draw, fill=blue!20, 
    text width=5em, text centered, rounded corners, minimum height=4em, line width=1.6pt]
\tikzstyle{wideblock} = [rectangle, draw, fill=blue!20, 
    text width=10em, text centered, rounded corners, minimum height=4em]
\tikzstyle{line} = [draw, -latex']
\tikzstyle{cloud} = [draw, ellipse,fill=red!20, node distance=3cm,
    minimum height=2em]
\tikzstyle{arrow} = [thick,->,>=stealth]
\tikzstyle{textbf} = [draw,rectangle,text width=5cm,text centered]
\usetikzlibrary{arrows}

\usetikzlibrary{shapes,fit} %use shapes library if you need ellipse
\usepackage{adjustbox}
%charts
\usepackage{pgfplots}
\pgfplotsset{compat=1.7}

%Figure
\newenvironment{Figure}
  {\par\medskip\noindent\minipage{\linewidth}}
  {\endminipage\par\medskip}
%\usepackage[style=base,figurename=Figure,font={small,it}]{caption}


%-------- End Additions by Daniel
%--------------------------------------------------------------------------
%--------------------------------------------------------------------------
\begin{document}
%
% paper title
% Titles are generally capitalized except for words such as a, an, and, as,
% at, but, by, for, in, nor, of, on, or, the, to and up, which are usually
% not capitalized unless they are the first or last word of the title.
% Linebreaks \\ can be used within to get better formatting as desired.
% Do not put math or special symbols in the title.
\title{Automated Analysis of Inference Attacks on Property Preserving Encryption in the Symbolic Model}
%
% author names and IEEE memberships
% note positions of commas and nonbreaking spaces ( ~ ) LaTeX will not break
% a structure at a ~ so this keeps an author's name from being broken across
% two lines.
% use \thanks{} to gain access to the first footnote area
% a separate \thanks must be used for each paragraph as LaTeX2e's \thanks
% was not built to handle multiple paragraphs
%

% \author{Michael~Shell,~\IEEEmembership{Member,~IEEE,}
%         John~Doe,~\IEEEmembership{Fellow,~OSA,}
%         and~Jane~Doe,~\IEEEmembership{Life~Fellow,~IEEE}% <-this % stops a space
% \thanks{M. Shell was with the Department
% of Electrical and Computer Engineering, Georgia Institute of Technology, Atlanta,
% GA, 30332 USA e-mail: (see http://www.michaelshell.org/contact.html).}% <-this % stops a space
% \thanks{J. Doe and J. Doe are with Anonymous University.}% <-this % stops a space
% \thanks{Manuscript received April 19, 2005; revised August 26, 2015.}}
\author{Daniel~Becker}
% note the % following the last \IEEEmembership and also \thanks - 
% these prevent an unwanted space from occurring between the last author name
% and the end of the author line. i.e., if you had this:
% 
% \author{....lastname \thanks{...} \thanks{...} }
%                     ^------------^------------^----Do not want these spaces!
%
% a space would be appended to the last name and could cause every name on that
% line to be shifted left slightly. This is one of those "LaTeX things". For
% instance, "\textbf{A} \textbf{B}" will typeset as "A B" not "AB". To get
% "AB" then you have to do: "\textbf{A}\textbf{B}"
% \thanks is no different in this regard, so shield the last } of each \thanks
% that ends a line with a % and do not let a space in before the next \thanks.
% Spaces after \IEEEmembership other than the last one are OK (and needed) as
% you are supposed to have spaces between the names. For what it is worth,
% this is a minor point as most people would not even notice if the said evil
% space somehow managed to creep in.



% The paper headers
%\markboth{Journal of \LaTeX\ Class Files,~Vol.~14, No.~8, August~2015}%
%{Shell \MakeLowercase{\textit{et al.}}: Bare Demo of IEEEtran.cls for IEEE Journals}
% The only time the second header will appear is for the odd numbered pages
% after the title page when using the twoside option.
% 
% *** Note that you probably will NOT want to include the author's ***
% *** name in the headers of peer review papers.                   ***
% You can use \ifCLASSOPTIONpeerreview for conditional compilation here if
% you desire.




% If you want to put a publisher's ID mark on the page you can do it like
% this:
%\IEEEpubid{0000--0000/00\$00.00~\copyright~2015 IEEE}
% Remember, if you use this you must call \IEEEpubidadjcol in the second
% column for its text to clear the IEEEpubid mark.



% use for special paper notices
%\IEEEspecialpapernotice{(Invited Paper)}




% make the title area
\maketitle

% As a general rule, do not put math, special symbols or citations
% in the abstract or keywords.
\begin{abstract}
%\todo[inline]{(100 words) Show	Conjecture	Verify	Propose Achieves	Benefits}
We present a viable, novel approach to detect inference attacks on cryptographic protocols using symbolic verification. The approach consists of a adding an auxiliary population model and inference rules to formal models of protocols. This preliminary work forms the basis for extending the utility of symbolic verification to a wider range of attack scenarios. The case studies in this paper demonstrate the utility of the approach as a tool for understanding inference attacks on Property Preserving Encryption.
\cite{InfrenceAttacks}
\end{abstract}

% Note that keywords are not normally used for peerreview papers.
\begin{IEEEkeywords}
 Cryptographic Protocol, Protocol verification, Symbolic Model, Property Preserving Encryption, Inference
\end{IEEEkeywords}






% For peer review papers, you can put extra information on the cover
% page as needed:
% \ifCLASSOPTIONpeerreview
% \begin{center} \bfseries EDICS Category: 3-BBND \end{center}
% \fi
%
% For peerreview papers, this IEEEtran command inserts a page break and
% creates the second title. It will be ignored for other modes.
\IEEEpeerreviewmaketitle

% \section{META}
% Using Tamarin can we model a property preserving encryption protocol and verify it in the symbolical model against a well specified threat profile. Can we get \tamarin{} to find the previously known attacks/limitations.

% \begin{itemize}
% \item Approach
% \item Select a suitable protocol
%   \item 	One that is implemnted somewhere rathr than just specified in a paper
%   \item 	One that has published attacks/limiations
%   \item 	One that is not too complex.
% \item Model evolution
%   \item 		Secure Channel between client and server
%   \item 		Deliberately reveal everything stored to sysadmin "Honest but Curious"
%   \item 		Prove secure with RND
%   \item 		Test with DET
%   \item 		Test with OPE
%   \item 		Test with Search
% \end{itemize}



% Candidate Protocol 1
% 	OPE from CryptDB, "attacked" in \cite{InfrenceAttacks}

% Candidate Protocol 2
% 	\textbf{DTE from CryptDB, "attacked" in \cite{InfrenceAttacks}}


% Notes:
% Distinction between a scheme and a protocol.. are we attacking a protocol or a scheme plus a psuedo protocol that we've invented.
% Insight: Can we use oracles in the attack to cover the gaps in Tamarin. 
% Challenge is the threshold level.
% DET is probably simplest.

% 	Paper approach
% \begin{enumerate}
% \item 		Write Methodology
% \item 		Review Canditate Protocols
% \item 		Write Threat model text
% \item 		Write Tamarin Model for protocol execution
% \item 		Write model for threate model
% \item 		Write results
% \item 		Rewrite methodology
% \item 		Write introduction
% \item 		Write conclusions
% \item 		Write Abstract
% \item 		Edit
% \end{enumerate}
\todo[inline]{TODO Style Guide: consistent use of tamarin, data-set, population and column}


\section{Introduction}
% ( ?/2000 words) 
In response to numerous breeches of computer systems the use of encryption technique in the storage of data has increased. Poor design and poor implementation of these techniques has let to false confidence in their effectiveness and additional breech's of trust.
Numerous tools and techniques have been developed to improve the state of the  art and the correct application of them has been used to strengthen the security of systems and data\cite{SOK2017}.  
Symbolic Verification is a proven(\cite{Meier2013},\cite{5GAKA},\cite{18XOR}) technique for detecting potential attack scenarios and other weaknesses in cryptographic protocols.
In symbolical modelling of cryptographic protocols messages are represented as terms in a term algebra. The protocol and actions are modelled by multiset rewriting rules that define transition between states. Security properties are defined by propositional statements about the traces of the those transitions. The solver uses a backwards search approach to prove or falsify those properties. This paper presents an approach to extend symbolic modelling techniques(\cite{Meier2013} and \cite{ObsEqvCCS15}) to detect inference based attacks by modelling auxiliary population histograms and inference based attacks.
\subsection{Motivation }
%(?/500 words)}
This work explores initial directions in how automatic symbolical verification can be extended to detect potential inference attacks in cryptographic protocols, in particular Property Preserving Encryption protocols. The practical application of cryptographic protocols can be viewed as an efficiency/privacy trade-off\cite{GenericAttacks}, where efficiency is a measure functionality and performance. Property Preserving Encryption and other encrypted search techniques\cite{SOK2017} are an area of research that seeks to find the appropriate balance between these three competing concerns. 

One technique, Deterministic Encryption (DET) i.e an encryption that always produces the same encrypted text for a given plain-text, forms part of many cryptographically protected database search systems\cite{SOK2017} such as CryptDB \cite{Popa2011}, CipherBase \cite{cipherbase}, Microsoft's Always Encrypted mode in SQLServer\cite{AlwaysEncrypted}, Google's Big Query\cite{BigQuery} and \cite{Bellare2007}. As a technique DET is useful as it preserving the equality relationships permitting efficient searching and facilitating JOIN type relationships in search queries. Though DET encryption is a useful construct for building encrypted databases as it performs well, does not suffer from significant message expansion, and  permits searching it s know to leak equality information \cite{SOK2017} but \cite{InfrenceAttacks} shows that with a suitable auxiliary data-set the individual values are recoverable via inference. 

Inference attacks are one of the  oldest\cite{Arab} cryptographic attack models. Consider a simple letter substitution cipher (Caesar's cipher) where 'D' replaces every 'A', 'E' replaces every 'B' etc. An attacker can generate a letter frequency histogram from an auxiliary data-set such as previous messages or even a general corpus. For the English language message the attacker could assume that the most used letter in the enciphered message corresponds to the letter 'E'. The attacker can continue in this manner using individual letter frequency or include more sophisticate analysis of two-letter frequencies for example.

The real-world effectiveness of inference attacks depends on the availability of a well-correlated auxiliary data-set. Work\cite{Ismal2012} on the design of inference attacks  shows that it is possible to recover information about encrypted data-sets and queries from data leakage combined with auxiliary data-sets such as statistical census data. Work in \cite{InfrenceAttacks} shows that information can be recovered for data encrypted with Deterministic Encryption (DET) and Order Preserving Encryption (OPE) with a high degree of accuracy; 60\%+ for DET and 80\%+ for OPE in certain cases.

The number of possible values in the dataset affects the success of such attacks. For example the recovery rate of the patients \textit{ Age} was only 10\% in 83\% of the hospitals \cite{InfrenceAttacks}; this was due to the large number of values and that multiple values have similar frequencies.

The security of such a system then relies on the \textit{absence} of a strongly correlated auxiliary data-set and on the size and distribution of the values. 

As proving such a auxiliary data-set does not exist is likely impossible or at the very least impractical at to know at design time, detecting that a system is vulnerable in the presence of particular auxiliary data-set motivates the remediation or replacement of such a system. Symbolic Verification is an established and proven technique for identifying problems with systems and this work extends the class of problems that can be detected and provides a tool to quantify to what extent systems are vulnerable.


\subsection{Threat Model}
This paper considers a threat model where an adversary has access to an encrypted data set (legitimately as in an honest-but-curious security model or illegitimately in a a breech scenario). 
The attacker also has access to an auxiliary dataset that is structured in the same manner as the encrypted dataset i.e. same data-types and number of distinct values. This auxiliary dataset could be in the form of a publicly available dataset or it could be the result of a side-channel attack \cite{Kocher96}. Examples of side channel attacks that might generate suitable auxiliary data include timing analysis attacks, power analysis attacks and traffic analysis attacks\cite{SideChannelSurvey}.
The goal of the adversary is to retrieve \textit{all} values in the unencrypted dataset; either because it is of direct value to them or it can be used as an additional Auxiliary dataset to attack another encrypted dataset. 
The attack is based on the attacker using that the frequency distribution of a column is similar to that of the population and concluding with some degree of confidence what the  given encrypted value is. 

The "quality" of an auxiliary dataset is application-dependant\cite{InfrenceAttacks}; an Auxiliary dataset that is strongly correlated with the target data can give an attacker good results but may not be available to an attacker. 

% \subsection{Inference}
% \todo[inline]{500 words}
% \todo[inline]{O: Islam, Kuzu and Kantrcioglu \cite{Ismal2012} showed that inference attacks are useful , summarise contribution.}


\subsection{Contribution (?/1000 words)}
We present a symbolic validation technique for encryption protocols in the face of inference attacks after \cite{InfrenceAttacks}. This approach consists of a method to inject auxiliary population data and associated histograms in to a symbolic verification model. The model also contains an \textit{Inference} rule that models the reveals of encrypted data when it can be inferred from comparing the histograms of the modelled encrypted dataset to the auxiliary population. This technique can be used as a basis for research into attacking other Property Preserving techniques.  We present a number of initial case studies applying the technique.

\section{Preliminaries}

\subsection{Symbolic Verification with \tamarin{}}
\tamarin{}\cite{Meier2013} is an automated symbolic verification tool for analysing cryptographic protocols. \tamarin{} also for the modelling of complex security properties, cryptographic primitives and state. It is an open-source\cite{TamarinGithub} (GPL 3.0 Licensed), actively maintained project with a rich community of contributors. The project is also a rich repository of models used in previous publications including analysis of the 5G AKA protocol\cite{5GAKA}, the DPN3  Power grid protocol\cite{DNP3} and Alethea\cite{Alethea} a secure voting system.

\subsection{Observational Equivalence}
If two systems appear the same to the outside Environment they are set to be \textit{Observational Equivalent}.
This definition can be used to specify properties of a protocol and also in verification of models and implementations. 
\cite{ObsEqvCCS15} demonstrates a definition of Observational Equivalence and uses \tamarin{} with a number of case studies to illustrate the effectiveness of the approach for proving security properties on protocols with mutable state and unbounded numbers of sessions. \tamarin{} \cite{Schmidt2012AutomatedProperties} prover is a widely used\cite{ARPKICCS14, Donenfeld} tool for symbolic verification in general and numerous case studies\cite{Norwegian,5GAKA} validate the utility of the Observational equivalence approach.

The automate proof uses a \textit{bi-system} approach. This approach is a multiset rewriting system with a \textit{diff} operator \cite{AutomatedObsEqv18}.  A system modeller can specify two similar systems, denoted left and right, which differ in some terms, with the goal of the automated proved prove that left and right are Observational Equivalent, i.e an attacker can not determine the difference between the two systems.
\section{Methodology}
\subsection{Implementing Inference Attack in a Symbolical Model with \tamarin{}}
%\todo[inline]{1500 words}
The attacker has access to an encrypted dataset \enc{} over the message space \encSpace{} and plaintext auxiliary dataset \aux{} over the message space \auxSpace. Using a frequency analysis approach similar to the one described in \cite{InfrenceAttacks} an attacker will assign plaintext values to to each of the encrypted values by rank ordering the dataset histograms. i.e the most common value in \enc{}\ is declared to be the most common value in \aux{} etc.

Models are encoded into state transition \texttt{rules} that model how the protocol proceeds from state to state. The model contains a single shared state for all parties in the protocol and as the protocol is executed \texttt{traces} are recorded of the states and transitions used. Security properties (Secrecy, Uniqueness etc) and execution properties (A must happen before B etc) are enforced via \texttt{lemmas}. Violation of lemmas leads to a proof being falsified.

When run in Observational Equivalence mode two versions of the same protocol system are evaluated with an additional set of Equivalence lemmas that verify if the two versions appear the same to an outside observer.

Figure \ref{fig:approach} provides a schematic of the relationship between the key elements of the approach used.

 By convention the \textbf{Right Hand System (RHS)} is the secure system while the \textbf{Left Hand System (LHS)} is the system that will be varied under test. Both systems contain the following elements:
\begin{enumerate}[label=(\Alph*)]
\item \textbf{Protocol Model} The protocol is modelled with \tamarin{} state transition rules. \tamarin{} uses these state transition rules to move between states. The models of the protocols under test contain \textbf{Modify Rules} that modify the encrypted dataset (i.e. insert new values). Both systems have identical modification rules.
\item \textbf{Security and Execution} properties are encoded in the form of \texttt{lemmas} in both systems. 
\begin{enumerate}
	\item \LSecurity{Confidentiality}: the adversary never knows the encrypted values \enc{}. If this lemma is violated the proof fails.
    \item \LExecution{Encryption and Decryption}: Lemma to ensure that has the protocol executed correctly. (A protocol that never executes remains secure and can give false positive results).
	\item \LExecution{Growth}: The dataset of encrypted values \enc{} grows correctly as new values are added and that the set of counters increments correctly as the population grows; these counters model the distribution in the protected data.
\end{enumerate}

\item \textbf{The Auxiliary Data-set \aux{} over \auxSpace}:
In the RHS the Auxiliary Data-set \auxsecure{} is an evenly distributed un-encrpted population.  The auxiliary population and its histogram data is encoded into the model using \tamarin{} multi-sets as \tamarin{} currently has no direct support for counters. 
In the LHS the distribution of the auxiliary population is varied and injected (along with the distribution histograms) in to the model using a script external to \tamarin{}.
\item\textbf{ Inference Based Reveal Rule}. This is the key element of this approach, it encodes the "intuitive leap" required to connect the distribution of the auxiliary population with the encrypted dataset.   If  distribution of \aux{} matches \enc{} the rule will reveal the plaintext value. The rule is the same in the LHS and RHS.
\item \textbf{Execution}
The \tamarin{} prover is executed with each of generated \aux{} populations in \textbf{Observational Equivalence} mode. The prover can vary the target encrypted dataset distribution by inserting encrypted values using the modification rules. If the prover can not distinguish between the LHS and RHS then we conclude that the protocol is secure for that auxiliary population distribution.  In the case of an evenly distributed auxiliary dataset the there is nothing to infer from regardless of the distribution of the target dataset. In the case of a non-evenly distributed auxiliary data-sets the attacker can attack a target data-set with the same distribution. Each permutation of the auxiliary dataset is compared with the evenly distributed dataset in this manner. If the two protocols analysis are observational equivalence an attacker can not reason about the differences and Tamarin can prove that an adversary can not distinguish between the two systems and we can prove that the data is secure in the presence of such an auxiliary dataset.

\end{enumerate}

\begin{figure}
 
%\begin{adjustbox}{width=\linewidth}
        \resizebox{\linewidth}{!}{
        % Define the layers to draw the diagram
            \pgfdeclarelayer{background}
            \pgfdeclarelayer{foreground}
            \pgfsetlayers{background,main,foreground}
            \begin{tikzpicture}[node distance = 3cm, auto]
                % Place nodes
                \node (L_aux00) [block] {};
                \path (L_aux00.south east)+(-.75, 0.5) node (L_aux01) [block]{};
                \path (L_aux01.south east)+(-.75, 0.5) node (L_aux02) [block]{};
                \path (L_aux02.south east)+(-.75, 0.5) node (L_aux) [thickblock]{ Auxiliary Data-sets $\aux{}_n$ over \auxSpace};
                
                
                \path (L_aux.east)+(2,0) node  (L_insert) [block] {Modify Rules};
                \path (L_insert.south)+(0,-2) node  (L_enc) [block] {Encrypted Data \enc{} over \encSpace{}};
                
              
                \path (L_insert.east)+(3,0) node  (R_aux) [thickblock] {Evenly Distributed Auxiliary Data-set \auxsecure};
              
                \path (L_insert.east)+(6,0)  node  (R_insert) [block] {Modify Rules};
                \path (R_insert.south)+(0,-2) node  (R_enc) [block] {Encrypted Data \enc{} over \encSpace{}};
              
                 
                \draw[arrow]  (L_insert.south) -- node [above]{}(L_enc.90);
                \draw[arrow]  (R_insert.south) -- node [above]{}(R_enc.90);
                
                \path (L_enc.south)+(-1.5,-2) node (L_infer) [block] {Infer Rule};
                \path (R_enc.south)+(-1.5,-2) node (R_infer) [block] {Infer Rule};
                
                \draw[arrow]  (L_aux.south) -- node [above]{}(L_infer.90);
                \draw[arrow]  (R_aux.south) -- node [above]{}(R_infer.90);
                
                \draw[arrow]  (L_enc.south) -- node [above]{}(L_infer.90);
                \draw[arrow]  (R_enc.south) -- node [above]{}(R_infer.90);
                
                \path(L_infer.south)+(0,-2) node (L_lemma) [block] {Lemmas};
                \path(R_infer.south)+(0,-2) node (R_lemma) [block] {Lemmas};
                
                \draw[arrow]  (L_infer.south) -- node [above]{}(L_lemma.90);
                \draw[arrow]  (R_infer.south) -- node [above]{}(R_lemma.90);
                
                \path (L_lemma.south)+(+3.5,-2) node (OBS) [wideblock] {Observationally Equivalent?};
             
                 \draw[arrow]  (L_lemma.south) -- node [above]{}(OBS.90);
                 \draw[arrow]  (R_lemma.south) -- node [above]{}(OBS.90);
             
                \
                 \begin{pgfonlayer}{background}
                        \path (L_aux00.west)+(-0.5,2) node (L_bg_nw) {};
                        \path (OBS.north -| L_enc.east)+(0.5,0.5) node (L_bg_se) {};
                          
                        \path[fill=yellow!20,rounded corners, draw=black!50, dashed]
                            (L_bg_nw) rectangle (L_bg_se) {};     
                            
                        \path (L_bg_nw.north) +(3.5,-1) node (L_title) {\textbf{Left Hand System}};
                    
                        \path (R_aux.west)+(-0.5,2.7) node (R_bg_nw) {};
                        \path (OBS.north -| R_enc.east)+(0.5, 0.5) node (R_bg_se) {};
                          
                        \path[fill=yellow!20,rounded corners, draw=black!50, dashed]
                            (R_bg_nw) rectangle (R_bg_se){};        
                    
                        \path (R_bg_nw.north)+ (3,-1) node (R_title) {\textbf{Right Hand System}};
 
                \end{pgfonlayer} 
            \end{tikzpicture}
            }

%\end{adjustbox}
\caption{ Schematic Overview of the use of method used. \tamarin{} is repeatedly used in Observational Equivalence mode to compare a Left Hand System containing a varying Auxiliary data-set $z_n$ to a Right Hand System containing an evenly distributed data-set $z_{secure}$. The systems, including lemmas and state transition rules for insertion and inference are identical in all other respects. Procedural exploration of the permutations of the auxiliary dataset allows identification insights in to the particular situations where a protocol is secure or not.}
 \label{fig:approach}
 \end{figure}

 
\section{Case Studies on applying technique}
\subsection{Attacking 2-value Deterministic Encryption ?/1500 words}
Deterministic encryption (DET) always encodes the same value the same way. It is well known that it is vulnerable to frequency analysis ( a technique used since the 9th century\cite{Arab} for breaking classical encryption techniques including DET). As mentioned above DET is a widely used construct in encrypted database systems. We consider a DET encrypted database column representing the sex of patients who stayed at a hospital; a simple histogram of the two possible values might show a 55\%/45\% split. An attacker could infer that the 55\% represents females as auxiliary data-sets show more of the patients will be female \cite{InfrenceAttacks} than male. This fact can then be leveraged to infer other attributes; i.e certain disease types affect females more than males; more males are born that females and females have a longer life expectancy. These publicly available statistics are an auxiliary dataset that an attacker can freely use. 

This protocol is modelled and attacked as follows:
\begin{enumerate}[label=(\Alph*)]
        \item A. The DET/2 protocol is encode as two pairs of equations which are used by the \textbf{Modify Rules}:
    \begin{enumerate}
        \item 	$DEnc(1) -> X,	DEnc(0) -> Y$
    	\item 	$DDec(X) ->  DEnc(1), DDec(Y) ->  DEnc(0)$
        \item       The actual X and Y values are determined by the setup phase are and do not affect the proof.
    \end{enumerate}
	When a value is stored using the DEnc equation used the appropriate histogram counters are also incremented. Tamarin multi-sets are used as a base-1 counter (i.e $'X'+'X'+'X'+'X'+'X' = 5$).  This approach increments the counters as data is inserted and was found to be significantly faster than the alternative approach of counting on demand.
\item The \aux{} population is encoded as un-ordered multi-sets (e.g $ 1' + '1' ... ' + '0' + '0'$) and the \aux{} histogram counters are injected as base-1 multi-sets.
\item The inference rule reveals the encrypted values of X and Y if the auxiliary population distribution uniquely matches the encrypted dataset population.
\item When executing Tamarin will execute each model twice, if it can not distinguish between the two cases (50/50 and the one under test) then the proof is valid. The population is varied as follows: 0/10\%, 10/90\% ..., 90/10\%, 100/0\%. 
Using the built in Observational Equivalence function diff(); Tamarin attempts to prove each model twice, once with the Aux population and once with a perfectly distributed set of values (i.e. 50/50\%). 
\item As a control the experiment is repeated using non-deterministic Random encryption with the expectation that the data is secure regardless of population distribution. This is discussed below.
\end{enumerate}
\subsubsection{Results} As intuitively expected only the protocol is only proved secure in the presence of a 50/50 auxiliary distribution. All other auxiliary distributions allow the symbolic verified to find an attack scenario via inference. This is an important validation of validity of the approach.
\todo[inline]{Add timings to prove it runs in a sensible amount of time}
\subsection{Attacking 2-value Random Encryption}
As a control for the above case we also modelled a similar protocol replacing the DET equations with non-deterministic encryption.
This mode of encryption does not allow for searching and is not classed as Property Preserving. This is modelled by the RND equation below. This form of encryption is typically implemented by including a fresh random value in the encryption and as a side effect this technique causes message expansion. This random value is discarded on decryption. A RND encryption function is used in \cite{Popa2011} and other systems as a secure but unsearchable encryption option.
\begin{enumerate}
    \item $rnd(key, rnd(key, m))=m$
\end{enumerate}
The execution method was identical to that used in the DET case above.
\subsubsection{Results}

As expected the prover was not able to find an inference attack in any auxiliary population scenario. \todo[inline]{Add timings to prove it runs in a sensible amount of time}

\subsection{Attacking Hash functions}
\todo[inline]{TODO Pretty trivial to rerun over a bunch of different hash functions which will be insecure. Little bit more to implement Hash + salt which will be secure.  Hashing is a form of DET and is commonly used. When used with out proper salt its vulnerable. (Could model this handily and find references to examples where it is not salted correctly and breached) }
\tamarin{} has builtin support for a hashing equation. This equation 

\subsection{Generalisation - Attacking Det without external Auxiliary data}
\todo[inline]{We ran models to confirm that the inference rule can generalise and find attacks without the presence of an specified auxiliary population. This approach is based on having the prover find a combination of encrypted data and auxiliary data that leads to data being revealed via the inference rule. This approach is useful for to find that a protocol is insecure but it does not help enumerate the conditions under which the inference is a problem. This approach may be particularly useful for finding attacks where the auxiliary data is a bi-product of the  protocol execution, i.e side channel attacks etc}
\section{Future Work}
Future work areas include generalising this technique and exploring adding inference rules automatically to existing symbolic models to extend the applicability of the technique. Investigations could be made to model a  ℓₚ attack\cite{InfrenceAttacks}. Currently the infer rule use an exact match, the rule could be extended to have reveal within a specified confidence level of a match.

%\todo[inline]{
% Consider how a  ℓₚ approach would work in Tamarin? very difficult with out maths.... %Could perhaps use python to generate a bunch of rules?
%}


%\todo[inline]{Re-run using different Tamarin heuristic models to see if performance ca be %improved. This is likely just stretching out the data but no real insights. The heuristics for diff mode are pretty limited anyway
%}
%\todo[inline]{That bit about more numbers starting with 1 that other numbers (is there a %reference for this)
%}

% \todo[inline]{ What remains unresolved? 
% 	The Tamarin tool could be extended to have the means to vary an Auxiliary population baked in to it. Other protocols could be explored; ARX for example or ORAMS. The statistical significance work has not been undertaking; we've merely declared some threshold exists.
% 	REmoval of False positives

% }

 \todo[inline]{
There is a model of attacking things link ARX which involved having access to a complete backup of the database and applying a Cash\cite{CashLeackageAbuse} like approach to infinitely querying it.
}

\subsection{Outline of proposed approach for 3 Value Det.}
\todo[inline]{A. Rules as above extended to 3 value (A,B,C)
B. Similar as above
C. The inference rule is build uses the following histograms:
		Count of A, count of Not A
		Count of B, count of Not B
		Count of C, count of Not C
The insight being that Not A is equivalent to  [B,C] and we can solve for A first then solve for B, C as previously.
Note that this approach is O(nCr) due to the modelling of the auxiliary databases. 
}

\todo[inline]{Example of a side channel attack? Their is a spthy in examples for a time sensitive protocol}
\section{Conclusion ?/200 words}
This novel approach of injecting a auxiliary population and inference rule into a symbolic verification model has proved feasible for some scenarios and is a promising basis for the analysis of other Property Preserving Encryption protocols. 	Symbolical verification can be used for a broader range of protocols; and help understand that certain protocols should not be used except in very carefully controlled environments.


% if have a single appendix:
%\appendix[Proof of the Zonklar Equations]
% or
%\appendix  % for no appendix heading
% do not use \section anymore after \appendix, only \section*
% is possibly needed

% use appendices with more than one appendix
% then use \section to start each appendix
% you must declare a \section before using any
% \subsection or using \label (\appendices by itself
% starts a section numbered zero.)
%


\appendices

\section{\tamarin{} Code fragments}
This appendix includes the key code sections of the  approach.

% In this code listing ¿ is used as an escape character for comments an latex anchors etc
\begin{lstlisting}[language=Tamarmin, escapechar=¿, 
caption={Example state transition rule for Auxiliary Dataset and Associated histogram. Breifly, \tamarin{} models protocols as \textit{multiset rewriting rules} which operate on the sysmtem state. This rule models the initialisation of an auxiliary dataset and a two counter histogram. Rules consist of a \textit{Premise(Line \ref{lst:CreateAuxDatasetRulePremise})} , \textit{Actions Facts(Line \ref{lst:CreateAuxDatasetRuleActionFacts})} and a \textit{Conclusion(Lines \ref{lst:CreateAuxDatasetRuleConclusionBegin}-\ref{lst:CreateAuxDatasetRuleConclusionEnd})}. The premise specifies what must be present before this rule can execute, tn this example the Premise only requires a \textit{fresh} variable which an be generated at any point by the built-in \textit{Fr} fact at any point. This example does not contain any Actions Facts; conceptually they are used to annotate the execution trace and can be reference from lemmas to reason about the execution. The conclusion represents the state after the rule is executed; this rule contains three facts:  the Auxiliary Population(Line \ref{lst:CreateAuxDatasetRuleConclusionFactA}), the 1's counter(Line \ref{lst:CreateAuxDatasetRuleConclusionFactB})   and the 0's counter (Line \ref{lst:CreateAuxDatasetRuleConclusionFactC}). All three of these rules contain the \texttt{diff} operator, which is used in Observational equivalence mode to select the appropriate facts for the Left Hand Side(LHS) and Right Hand Side(RHS) systems. In this example the Auxiliary population  for the LHS is contains 30\% '1's and 70\% '0's, the RHS contains the equally distributed 50\%/50\% population. The two histogram counters are modelled as base-1 un-ordered multi-sets; theses counters model the distribution of the auxiliary populations and are compared to the target population in the inference rule. The 'ZERO' value is required for consistency when modelling empty multi-sets.},label={lst:CreateAuxDataset}]
rule CreateAuxDataset:  ¿\label{lst:CreateAuxDatasetRule}¿
 [ Fr(~p)] ¿\label{lst:CreateAuxDatasetRulePremise}¿
 --[]-> ¿\label{lst:CreateAuxDatasetRuleActionFacts}¿
   [AuxDataset(~p, ¿\label{lst:CreateAuxDatasetRuleConclusionBegin}¿
        diff( '1'+'1'+'1'+'0'+'0'+'0'+'0'+'0'+'0'+'0',  ¿\label{lst:CreateAuxDatasetRuleConclusionFactA}¿
              '1'+'1'+'1'+'1'+'1'+'0'+'0'+'0'+'0'+'0')),
    AuxDataset1Counter(~p,  ¿\label{lst:CreateAuxDatasetRuleConclusionFactB}¿
        diff( 'ZERO'+'X'+'X'+'X','ZERO'+'X'+'X'+'X'+'X'+'X')),  
    AuxDataset0Counter(~p,  ¿\label{lst:CreateAuxDatasetRuleConclusionFactC}¿
        diff( 'ZERO'+'X'+'X'+'X'+'X'+'X'+'X'+'X','ZERO'+'X'+'X'+'X'+'X'+'X'))
    ] ¿\label{lst:CreateAuxDatasetRuleConclusionEnd}¿

\end{lstlisting}

% In this code listing ¿ is used as an escape character for comments an latex anchors etc
\begin{lstlisting}[language=Tamarin, escapechar=¿, 
caption={Example state transition rule for Inference. Informally, this rule reveals the encrypted values when the histograms of the encrypted data and the auxiliary data match. This example shows two different techniques to model the histogram; \texttt{EncData}(Line \ref{lst:InferRuleEnc}) models the data and the histogram counters as part of a single fact while the auxiliary histogram counters are encoded as two separate facts(Lines \ref{lst:InferRuleCounter0} and \ref{lst:InferRuleCounter1}). Combining the counters in a single fact reduces the verification search space and speeds up proofs. The premise of this example rule(Lines  \ref{lst:InferRulePremiseStart}-\ref{lst:InferRulePremiseEnd}) specifies that this rule may only be executed when \texttt{aval} and \texttt{bval} are equal in all the counters i.e. the histograms for the encrypted and auxiliary data-sets match exactly. This rule has an \texttt{Action Fact}(Line \ref{lst:InferRuleActionFacts}) that is used in the the security property lemmas to indicate that the data is no longer secret. The conclusion of this example(Line \ref{lst:InferRuleConclusion}) uses the built-in \texttt{K()} fact to model that the adversary now knows these values through inference. } ,label={lst:InferRule}]
rule Infer:
   [¿\label{lst:InferRulePremiseStart}¿
   EncData( val, aval, bval ), ¿\label{lst:InferRuleEnc}¿
   AuxDataset0Counter( ~p, aval ),¿\label{lst:InferRuleCounter0}¿
   AuxDataset1Counter( ~p, bval ) ¿\label{lst:InferRuleCounter1}¿
   ]¿\label{lst:InferRulePremiseEnd}¿ 
  --[ Revealed( val ) ]-> ¿\label{lst:InferRuleActionFacts}¿
   [ K( '1' ), K( '0' ) ] ¿\label{lst:InferRuleConclusion}¿

\end{lstlisting}


%\lstinputlisting[language=Tamarmin]{models/AttackingDET_6040.spthy}





% use section* for acknowledgment
%\section*{Acknowledgment}
%The authors would like to thank...


% Can use something like this to put references on a page
% by themselves when using endfloat and the captionsoff option.
\ifCLASSOPTIONcaptionsoff
   \newpage
\fi
% % \todo[inline]{
% Browns 8 Questions:
% 1. Who are intended readers? (3-5 names) 
% Someone like Cas Cremer from the the tamarin world
% Charles V Wright from the inference paper (and helped with the SoK paper)
% Someone like Popa or other protocol developer

% 2. What did you do? (50 words)
% 	1. Developed a set of machine verifiable symbolic models encompassing a property preserving encryption protocol, auxiliary population and set of security properties to be verified.
% 		The model includes a specification for the attacker model and an new inference rule.
% 	2. Verify the model using Tamarin with a set of auxiliary populations where the distribution is varied.

% 3. Why did you do it? (50 words)
% 	Encryption protocols are hard to get right. Extending the symbolical model to include inference attacks extends the capabilities of the tooling and allows additional attack vectors to be found. This work does not seek to generate new attack models but to automated the discovery tof potential attacks in proofs.
% 4. What happened? (50 words)
% 	The tooling was successfully extended to iterate over multiple auxiliary populations demonstrating the importance of introducing auxiliary data in to models and the viability of the approach. The results on theoretical models match well with empirical results previously published in /cite{inference}. The code used in the approach was published to allow others to extend on this work.
% 5. What do results mean in theory? (50 words)
% 	This approach extends the the applicability of formal validation of cryptographic protocols. It clearly highlights some boundaries for using PPE techniques. (TODO: Does this mean we can say never use DET if less that N distinct values?). 

% 6. What do results mean in practice? (50 words)
% 	Symbolical verification can be used for a broader range of protocols; certain protocols should not be used excerpt in very carefully controlled environments. Auxiliary populations are very powerful.
% 7. What is the key benefit for readers (25 words)
% 	The reader understands how the availability and use of an auxiliary populations adds an attacker. How symbolical verification with an auxiliary population can reveal additional attack vectors. 
% 8. What remains unresolved? (no word limit) 
% 	The Tamarin tool could be extended to have the means to vary an Auxiliary population baked in to it. Other protocols could be explored; ARX for example or ORAMS. The statistical significance work has not been undertaking; we've merely declared some threshold exists.


 

% }


% trigger a \newpage just before the given reference
% number - used to balance the columns on the last page
% adjust value as needed - may need to be readjusted if
% the document is modified later
%\IEEEtriggeratref{8}
% The "triggered" command can be changed if desired:
%\IEEEtriggercmd{\enlargethispage{-5in}}

% references section

% can use a bibliography generated by BibTeX as a .bbl file
% BibTeX documentation can be easily obtained at:
% http://mirror.ctan.org/biblio/bibtex/contrib/doc/
% The IEEEtran BibTeX style support page is at:
% http://www.michaelshell.org/tex/ieeetran/bibtex/
%\bibliographystyle{IEEEtran}
% argument is your BibTeX string definitions and bibliography database(s)
%\bibliography{IEEEabrv,../bib/paper}
%
% <OR> manually copy in the resultant .bbl file
% set second argument of \begin to the number of references
% (used to reserve space for the reference number labels box)

\bibliographystyle{IEEEtran}
\bibliography{bibtex/references}


% biography section
% 
% If you have an EPS/PDF photo (graphicx package needed) extra braces are
% needed around the contents of the optional argument to biography to prevent
% the LaTeX parser from getting confused when it sees the complicated
% \includegraphics command within an optional argument. (You could create
% your own custom macro containing the \includegraphics command to make things
% simpler here.)
%\begin{IEEEbiography}[{\includegraphics[width=1in,height=1.25in,clip,keepaspectratio]{mshell}}]{Michael Shell}
% or if you just want to reserve a space for a photo:

% \begin{IEEEbiography}{Michael Shell}
% Biography text here.
% \end{IEEEbiography}

% % if you will not have a photo at all:
% \begin{IEEEbiographynophoto}{John Doe}
% Biography text here.
% \end{IEEEbiographynophoto}

% % insert where needed to balance the two columns on the last page with
% % biographies
% %\newpage

% \begin{IEEEbiographynophoto}{Jane Doe}
% Biography text here.
% \end{IEEEbiographynophoto}

% You can push biographies down or up by placing
% a \vfill before or after them. The appropriate
% use of \vfill depends on what kind of text is
% on the last page and whether or not the columns
% are being equalized.

%\vfill

% Can be used to pull up biographies so that the bottom of the last one
% is flush with the other column.
%\enlargethispage{-5in}



% that's all folks

\end{document}


